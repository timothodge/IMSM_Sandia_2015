\documentclass[10pt]{article}
\usepackage[final]{graphicx}
\usepackage{amsfonts}

\topmargin-.5in
\textwidth6.6in
\textheight9in
\oddsidemargin0in

\def\ds{\displaystyle}
\def\d{\partial}

\begin{document}

\centerline{\large \bf Given a helical compression spring in a spring-mass-damper system, what are optimal springs?}

\vspace{.1truein}

\def\thefootnote{\arabic{footnote}}
\begin{center}
  Justin Krueger\footnote{Mathematics, Virginia Tech University},
  Alistair Bentley\footnote{Mathematics, Clemson University},
  Tianyu Qiu\footnote{Mathematics, University of Delaware},
  Saideep Nannapaneni\footnote{Civil & Environmental Engineering,Vanderbilt University},
  Jiahua Jiang\footnote{Mathematics, University of Massachusetts Dartmouth },
  Tim Hodges\footnote{Mathematics, Colorado State University}
\end{center}

%\vspace{.1truein}

\begin{center}
Faculty Mentors: Mentor 1\footnote{Company},
Mentor 2\footnote{University}
\end{center}


\vspace{.3truein}
\centerline{\bf Abstract}


An optimal spring depends on a set of constraints and objectives. The application of an optimal spring is subject to change, and for this reason it is important to allow constraints to become objectives and vice versa. The objective of this project is to implement a flexible way to compute an optimal spring given a set of constraints and objectives that are subject to change. 


\begin{itemize}
\item Summarize the results presented in the report, and the contributions
of your research.




\item Readers should not have to look at the rest of the paper in order to 
understand the abstract.

\item Keep it short and to the point.
\end{itemize}

\section{Introduction}
%It should be written as much as possible in non-technical terms, so that a
%lay reader can understand the context and the contribution of the paper.

The design and 

\begin{itemize}
\item Describe the problem you are trying to solve, the approach
you took, and summarize your contribution and results.

\item Review the history of this problem, and existing literature.

\item Give an outline of the rest of the paper.
\end{itemize}

\section{The Problem}
\begin{itemize}
\item Give a precise technical description of your problem. 

\item State and justify all your assumptions. 

\item Define notation. 

\item Describe your data, how you collected them, their properties,
and whether you did 
anything to them (removed noise, filled in missing data, 
applied normalizations).
\end{itemize}

\section{The Approach}
\begin{itemize}
\item Present and justify your approach for solving the problem. 
\item Explain the advantages of your approach over existing ones.

\item Tell a story.
Don't just say: ``I did this, then I did this, and at last I did this''.
\end{itemize}

\section{Computational Experiments}
Give enough details so that readers can duplicate your experiments.

\begin{itemize}
\item Describe the precise purpose of the experiments, and what they 
are supposed to show.

\item Describe and justify your test data, and any assumptions you made to 
simplify the problem.

\item Describe the software you used, and the 
parameter values you selected.

\item 
For every figure, describe the meaning and units of the coordinate axes, 
and what is being plotted.

\item Describe the conclusions you can draw from your experiments
\end{itemize}

\section{Summary and Future Work}
\begin{itemize}
\item Briefly summarize your contributions, and their possible
impact on the field (but don't just repeat the abstract or introduction).
\item Identify the limitations of your approach.
\item Suggest improvements for future work.
\item Outline open problems.
\end{itemize}



\end{document}

